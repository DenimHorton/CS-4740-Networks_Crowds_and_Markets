\documentclass[11pt]{article}  
\usepackage[margin=1in]{geometry}
\parindent=0in
\parskip=8pt
\usepackage{fancyhdr,amssymb,amsmath, graphicx, listings,float,subfig,enumerate,epstopdf,color,multirow,setspace,bm,textcomp}
\usepackage[usenames,dvipsnames]{xcolor}
\usepackage{hyperref}
\usepackage{graphicx}
\graphicspath{{./Images}}

\pagestyle{fancy}


\begin{document} 

\lhead{Assignment \# 2}
\chead{Robert Denim Horton}
\rhead{\today}

\begin{center}\begin{Large}
CS 4720/5720 Design and Analysis of Algorithms

Homework \#2

Student: (Robert Denim Horton)
\end{Large}
\end{center}


\section*{Answers to homework problems:}

\begin{enumerate}
% Question 1
\item Consider the following graphs in Figure 1.1
\begin{center}
\includegraphics[scale=0.6]{Question1_Figure1.1}\\
Figure 1.1
\end{center}
	\begin{enumerate}[(a)]
		% Question 1: Part 1
		\item Are these graphs strongly connected? Explain why or why not.
		% Question 1: Part 2
		\item Are these graphs aperiodic? If a graph is periodic, compute its period.
	\end{enumerate}
% Question 2
\item Consider the following weighted adjacency matrix in Figure 1.2:
\begin{center}
\includegraphics[scale=0.6]{Question2_Figure1.2}\\
Figure 1.2
\end{center}
	\begin{enumerate}[(a)]
		% Question 2: Part 1
		\item Is this matrix row-stochastic?
		% Question 2: Part 2
		\item Draw the graph corresponding to this matrix.
		% Question 2: Part 3
		\item Is the graph strongly connected?
		% Question 2: Part 4
		\item Is the graph aperiodic?
	\end{enumerate}
% Question 3
\item Consider the following graph in Figure 1.3
\begin{center}
\includegraphics[scale=0.6]{Question3_Figure1.3}\\
Figure 1.3
\end{center}
	\begin{enumerate}[(a)]
		% Question 3: Part 1
		\item Write the weighted adjacency matrix corresponding to this graph. Verify that the graph you wrote is row-stochastic.
		% Question 3: Part 2
		\item By hand, using matrix multiplication, perform 1 step of the DeGroot opinion dynamic model on this graph with initial opinions of (1, 1, 0.5, 0, 0).
		% Question 3: Part 3
		\item In the DeGroot opinion dynamics model, which node's initial opinion gets the most weight in society's final opinion? Which node gets the least weight? (it is highly recommended that you use computer code to answer this question).
	\end{enumerate}
% Question 4
\item Consider again the graph from problem 2 (Figure 1.2).
	\begin{enumerate}[(a)]
		% Question 4: Part 1
		\item By hand, using matrix multiplication, perform 1 step of the Friedkin-Johnsen opnion dynamic model on this graph with initial opinions of (1, 1, 0.5, 0, 0) and lambda values of (0.9, 0.1, 0.8, 1, 0.5).
		% Question 4: Part 2
		\item As we discussed in class, in the Friedkin-Johnsen model, nodes usually have differing opinions even after a long time. Keeping the lambda values from part A, experiment with this graph and try to find an assignment of initial opinions that results in the most different limiting opinions. That is, what equilibrium on this graph can have the most disagreement out of all equilibria? You will likely want to use compute code to solve this problem. Full credit is possible if you show understanding and effort; I will award 1 bonus point if you can show that your answer is the most disagreement possible.
	\end{enumerate}
\end{enumerate}
\end{document}
