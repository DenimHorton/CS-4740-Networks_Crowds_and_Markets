\documentclass[11pt]{article}  
\usepackage[margin=1in]{geometry}
\parindent=0in
\parskip=8pt
\usepackage{fancyhdr,amssymb,amsmath, graphicx, listings,float,subfig,enumerate,epstopdf,color,multirow,setspace,bm,textcomp}
\usepackage[usenames,dvipsnames]{xcolor}
\usepackage{hyperref}
\usepackage{graphicx}
\graphicspath{{./Images}}

\pagestyle{fancy}


\begin{document} 

\lhead{Assignment \# 7}
\chead{Robert Denim Horton}
\rhead{\today}

\begin{center}\begin{Large}
CS 4720 Networks, Crowds, and Markets

Homework \#  7

Student: (Robert Denim Horton)
\end{Large}
\end{center}

\section*{Chapter 9}
\begin{enumerate}

	% Question 1
	 \item In this question we will consider an auction in which there is one seller who wants to sell one unit of a good and a group of bidders who are each interested in purchasing the good. The seller will run a sealed-bid, second-price auction. Your firm will bid in the auction, but it does not know for sure how many other bidders will participate in the auction. There will be either two or three other bidders in addition to your firm. All bidders have independent, private values for the good. Your firm’s value for the good is c. What bid should your firm submit, and how does it depend on the number of other bidders who show up? Give a brief (1-3 sentence) explanation for your answer 

	% Question 7
	\item  seller announces that he will sell a case of rare wine using a sealed-bid, second-price auction. A group of I individuals plan to bid on this case of wine. Each bidder is interested in the wine for his or her personal consumption; the bidders’ consumption values for the wine may differ, but they don’t plan to resell the wine. So we will view their values for the wine as independent, private values (as in Chapter 9). You are one of these bidders; in particular, you are bidder number i and your value for the wine is $V_i$.\\\\
How should you bid in each of the following situations? In each case, provide an explanation for your answer; a formal proof is not necessary.

\end{enumerate}

\section*{Chapter 10}

\begin{enumerate}
	% Question 2
	\item  Suppose we have a set of 3 sellers labeled a, b, and c, and a set of 3 buyers labeled x, y, and z. Each seller is offering a distinct house for sale, and the valuations of the buyers for the houses are as follows.
	% Uncomment to insert picture from Homeworkxx/Images
	%\includegraphics[scale=2.4]{Insert image name here no .png needed}
Suppose that sellers a and b each charge 2, and seller c charges 1. Is this set of prices
market-clearing? Give a brief explanation.

		
	% Question 3
	\item Suppose we have a set of 3 sellers labeled a, b, and c, and a set of 3 buyers labeled x, y, and z. Each seller is offering a distinct house for sale, and the valuations of the buyers for the houses are as follows.
	% Uncomment to insert picture from Homeworkxx/Images
	%\includegraphics[scale=2.4]{Insert image name here no .png needed}
Suppose that sellers a and c each charge 1, and seller b charges 3. Is this set of prices market-clearing? Give a brief explanation.

	
	% Question 4
 \item Suppose we have a set of 3 sellers labeled a, b, and c, and a set of 3 buyers labeled x, y, and z. Each seller is offering a distinct house for sale, and the valuations of the buyers for the houses are as follows.
	% Uncomment to insert picture from Homeworkxx/Images
	%\includegraphics[scale=2.4]{Insert image name here no .png needed}
Suppose that a charges a price of 3 for his house, b charges a price of 1 for his house, and c charges a price of 0. Is this set of prices market-clearing? If so, explain which buyer you would expect to get which house; if not, say which seller or sellers should raise their price(s) in the next round of the bipartite-graph auction procedure from Chapter 10.

\end{enumerate}
\end{document}
